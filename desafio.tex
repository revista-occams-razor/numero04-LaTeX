% Este fichero es parte del N�mero 4 de la Revista Occam's Razor
% Revista Occam's Razor N�mero 4
%
% (c)  2009, The Occam's Razor Team
%
% Esta obra est� bajo una licencia Reconocimiento 3.0 Espa�a de
% Creative Commons. Para ver una copia de esta licencia, visite
% http://creativecommons.org/licenses/by/3.0/es/ o envie una carta a
% Creative Commons, 171 Second Street, Suite 300, San Francisco,
% California 94105, USA. 

% Desaf�o 
%

\rput(5.0,-3.0){\resizebox{16cm}{!}{{\epsfbox{images/general/header.eps}}}}

% -------------------------------------------------
% Cabecera
\begin{flushright}
\msection{introcolor}{black}{0.25}{DESAF�O}

\vspace{4.5mm}

\mtitle{10cm}{Un Desaf�o Criptogr�fico}

{\sf por Fernando Mart�n Rodr�guez (fmartin@tsc.uvigo.es) }

{\psset{linecolor=black,linestyle=dotted}\psline(-12,0)}
\end{flushright}

\vspace{2mm}

% -------------------------------------------------


Despu�s de escribir el art�culo sobre criptograf�a me  entraron  unas  ganas
irresistibles de ponerme a inventar m�todos. Hablando con el director de  la
revista se nos ocurri� esta idea: publicar  un  desaf�o.  Esto  es,  lo  que
sigue es un algoritmo de criptograf�a de nuestra invenci�n. Tan  p�blico  va
a ser que os vamos a dar el c�digo fuente. Despu�s va un  criptograma  y  se
trata de descifrarlo... sin saber la clave claro.


\vspace{2mm}


En el pr�ximo n�mero se  publicar�  la  soluci�n  y  los  nombres  de  todos
aqu�llos que hayan enviado soluciones v�lidas.  Para  enviar  una  soluci�n,
enviad un correo a occams-razor@uvigo.es con:

\begin{itemize}
\item Nombre completo.
\item Criptograma descifrado.
\item Valor de la clave.
\item Una explicaci�n breve de lo que hace el m�todo y  de  c�mo  lo
hab�is atacado.
\end{itemize}

�nimo, con las pistas dadas, no es dif�cil.

El cifrador son dos funciones de matlab... ah� van (sin comentarios):
 

%\sectiontext{white}{black}{EL C�DIGO}

Funci�n principal:

\lstset{language=Matlab,frame=tb,framesep=5pt,basicstyle=\footnotesize}   
\begin{lstlisting}

function criptograma = CifrarTexto(llano,clave)
raiz = floor(sqrt(clave));
resto = clave-raiz^2;
N = length(llano);
for i=1:N,
    [raiz, resto, digito] = DigitoDecimalRaizCuadrada(raiz,resto);
    cfr = llano(i)+digito;
    if cfr>90
        cfr = (cfr-91)+65;
    end
    criptograma(i) = char(cfr);
end
\end{lstlisting}

Funci�n auxiliar:

\lstset{language=Matlab,frame=tb,framesep=5pt,basicstyle=\footnotesize}   
\begin{lstlisting}

function [raiz, resto, digito] = DigitoDecimalRaizCuadrada(raiz0,resto0)
% Obtener el siguiente decimal de una raiz cuadrada
resto1 = resto0*10;
resto2 = resto1*10;
aux1 = 2*raiz0;
digito0 = floor(resto1/aux1);
for cont = digito0:-1:0,
    aux2 = aux1*10+cont;
    aux3 = aux2*cont;
    resto = resto2-aux3;
    if (resto>=0)
        digito = cont;
        break;
    end
end
raiz = raiz0*10+digito;
\end{lstlisting}

���Aviso!!! No he ``encriptado'' los  nombres  de  las  funciones  ni  de  las
variables. No os doy el programa que hace el descifrado. �Podr�ais crearlo?

Y ahora el criptograma:

\begin{verbatim}
DWSGNGGJBVVPIWYXXKJSGHVLLPXVHQYWYAJBTLEDNLAROLXTBGINPFZYJRPXUMUFCVCJTMXRDKXIFUXRW
\end{verbatim}

Por �ltimo, para comprender totalmente el m�todo estar�a bien leer
esto:

\url{http://platea.pntic.mec.es/anunezca/ayudas/algoritmo_raiz/algoritmo_raiz.htm}
\EOP

%\sectiontext{white}{black}{FILTRANDO PAQUETES}


\clearpage
\pagebreak
