% Este fichero es parte del N�mero 4 de la Revista Occam's Razor
% Revista Occam's Razor N�mero 4
%
% (c)  2009, The Occam's Razor Team
%
% Esta obra est� bajo una licencia Reconocimiento 3.0 Espa�a de
% Creative Commons. Para ver una copia de esta licencia, visite
% http://creativecommons.org/licenses/by/3.0/es/ o envie una carta a
% Creative Commons, 171 Second Street, Suite 300, San Francisco,
% California 94105, USA. 

% Seccion El Rinc�n de los Lectores
%
% Incluye imagen del art�culo
\rput(1.8,-2.0){\resizebox{!}{6.0cm}{{\epsfbox{images/general/lectores.eps}}}}

% -------------------------------------------------
% Cabecera
\begin{flushright}
\msection{introcolor}{black}{0.35}{RINC�N DE LOS LECTORES}

\mtitle{7cm}{El Rinc�n de los lectores}

\msubtitle{10cm}{Vuestros comentarios, sugerencias,...}

{\sf por The Occam's Razor Team}

{\psset{linecolor=black,linestyle=dotted}\psline(-12,0)}
\end{flushright}

\vspace{2mm}
% -------------------------------------------------

\begin{multicols}{2}

\sectiontext{white}{black}{Erratas}

{\bf enviado por Roberto Gonz�lez Cardenete}
\vspace{2mm}
\hrule
\vspace{2mm}

\medskip

{\em Hola,

no s� si esta errata os la han comunicado ya. Es en el tercer n�mero,
art�culo sobre inteligencia artificial y aprendizaje m�quina, tabla
3. Para la previsi�n "lluvioso" el error es 2/5 y para "nublado" es
0/4. En la tabla aparece de forma opuesta, 0/4 para "lluvioso" y 2/5
para "nublado. 

De igual manera ocurre para el atributo "Hace viento". La regla "SI"
tiene un error de 3/6 y la regla "NO" de 2/8.

Enhorabuena por la revista y muchas gracias: es un placer leerla.

Un saludo,}

--

Muchas gracias Roberto por apuntarnos esta errata. Ya le hemos
incluido en la F� De Erratas del n�mero 3!!

\vspace{3mm}


\sectiontext{white}{black}{Otra Utilidad de vim}

{\bf enviado por Luis Rodr�guez}
\vspace{2mm}
\hrule
\vspace{2mm}

\medskip

{\em 
Se os ha olvidado comentar, como utilidad para programadores, que si
presionamos la tecla "\%" en modo comando sobre un \{,( o [, el cursor
se nos mover� autom�ticamente hacia el correspondiente \},),], lo cual
es tremendamente �til a la hora de depurar programas que utilizan esta
sintaxis de llaves, par�ntesis o corchetes para anidar estructuras de
control, ya que un �tem de estos sin emparejar no se mover�. 

Un saludo,}

--

Totalmente de acuerdo, una muy �til funcionalidad de vim. Como os
coment�bamos en el art�culo, hay cientos de interesantes comandos
ofrecidos por vim, como este que nos comenta nuestro amigo
Luis. Simplemente ejecutad el comando help (ya sab�is ESC : help) y a
leer!.

\vspace{3mm}

\columnbreak

\sectiontext{white}{black}{Fuentes \LaTeX}

{\bf enviado por muchos de vosotros}
\vspace{2mm}
\hrule
\vspace{2mm}

\medskip

Muchos de vosotros nos hab�is escrito solicitando las fuentes de todos
los n�meros de Occam's Razor. Eso es lo que dec�amos en la web. Como
ya sabr�is tenemos algunos problemas de espacio y hasta ahora solo nos
ha sido posible mantener las fuentes de la �ltima revista, puesto que
los ficheros de fuentes son bastante voluminosos.

Queremos agradecer a todos los que os hab�is ofrecido para hacer un
mirror de estos ficheros y comunicaros que a partir de este n�mero
todos los ficheros estar�n accesibles para descargar.

Todos los que todav�a est�is interesados en hacer un mirror de estos
ficheros pod�is poneros en contacto con nosotros de nuevo (para que no
se nos despiste nadie entre todos los mails archivados) y en la p�gina
oficial de la revista ({\footnotesize\url{http://wwebs.uvigo.es/occams-razor}})mantendremos una lista con todos estos mirrors y
el tipo de acceso que proporciona.

\vspace{3mm}


\sectiontext{white}{black}{Felicitaciones}

{\bf enviado por muchos de vosotros}
\vspace{2mm}
\hrule
\vspace{2mm}

\medskip

Seguimos recibiendo y agradeciendo todas vuestras felicitaciones que
nos animan a continuar con esta vuestra revista, aunque a veces sea un
poco a trompicones. Intentamos que esta secci�n, al igual que todas
las dem�s de la revista os resulte �til y por esa raz�n no incluimos
todos estos mensajes, que aunque muy necesarios para nosotros.


En cualquier caso. 

{\textsf{\textbf{MUCHAS GRACIAS POR VUESTRO APOYO!!}}}

%\vspace{3mm}



\vspace{6mm}

\colorbox{excolor}{
\begin{minipage}{.9\linewidth}
{\bf\sf\Large ENVIADNOS...}
\vspace{1mm}
\hrule
\bigskip

Vuestros comentarios, sugerencias, ideas, cr�ticas (constructivas
claro), correcciones, soluciones, disoluciones o cualquier cosa que se
os ocurra... a:

\bigskip

{\tt occams-razor@uvigo.es}

\bigskip

{\sf{\textbf{LOS ESPERAMOS!!!!}}}

\bigskip

\end{minipage}
}





\raggedcolumns
\pagebreak

\end{multicols}

\pagebreak
