% Este fichero es parte del N�mero 4 de la Revista Occam's Razor
% Revista Occam's Razor N�mero 4
%
% (c)  2009, The Occam's Razor Team
%
% Esta obra est� bajo una licencia Reconocimiento 3.0 Espa�a de
% Creative Commons. Para ver una copia de esta licencia, visite
% http://creativecommons.org/licenses/by/3.0/es/ o envie una carta a
% Creative Commons, 171 Second Street, Suite 300, San Francisco,
% California 94105, USA. 

% Secci�n Trucos


\rput(3,-2.5){\resizebox{!}{6cm}{{\epsfbox{images/general/varita3.eps}}}}
\begin{flushright}
\msection{red}{black}{0.1}{TRUCOS}

\mtitle{6cm}{Con un par... de l�neas}

\msubtitle{8cm}{Chuletillas para hacer cosas m� r�pido}

{\sf por Tamariz el de la Perd�z}

{\psset{linecolor=black,linestyle=dotted}\psline(-10,0)}

\end{flushright}

\vspace{4mm}

\begin{multicols}{2}
\raggedcolumns


\sectiontext{white}{black}{GDB CON INTERFAZ MAS AMIGABLE}
\hrule
\vspace{2mm}

El depurador de GNU gdb, proporciona un interfaz {\em m�s amigable},
si es que eso es posible. Para echarle un ojo, solo ten�is que utilizar la
opci�n {\tt -tui} (algo as� como {\em Text User Interface}). Lo que obtendr�is es algo como esto:

\myfig{0}{images/trucos/gdb_tui.eps}{0.8}
\vspace{2mm}

Os adelantamos que, este interfaz se suele ir a paseo en cuanto
nuestro programa empieza a escribir cosas en la consola. Si necesit�is algo todav�a m�s amigable pod�is probar {\tt cgdb}.

\myfig{0}{images/trucos/cgdb.eps}{0.8}

\sectiontext{white}{black}{DEPURANDO CON GDB Y LIBTOOL}
\hrule
\vspace{2mm}

Si utiliz�is la autotools de GNU para compilar una aplicaci�n que
incluye librer�as din�micas, habr�is observado que el supuesto
ejecutable que generan las herramientas, es realmente un {\em
script}. Si intent�is utilizar {\tt gdb} con ese {\em script},
obviamente, solo conseguir�is un bonito mensaje de error.

Para poder depurar nuestro programa, tenemos que utilizar {\tt
libtool} de la siguiente manera.


{\small
\begin{verbatim}
$ libtool --mode=execute gdb ./mi_ejecutable
\end{verbatim}
}

\columnbreak

\sectiontext{white}{black}{HELPDESK CON SCREEN}
\hrule
\vspace{2mm}

Seguro que alguna vez hab�is estado al tel�fono un mont�n de tiempo
dictando comandos a alguien en el otro lado. Al mismo tiempo, vuestro
interlocutor os va diciendo que es esa cosa que hace que no le
funciona (porque falta un espacio, porque no ha pulsado ENTER, ...). 

La utilidad {\tt screen} puede hacernos la vida m�s f�cil en estas
situaciones. El modo multiusuario nos permite conectar dos instancias
de {\tt screen}, de forma que podremos ver lo que el otro usuario
escribe y nuestro cliente podr� ver que es lo que nosotros escribimos.

Para conectarnos a una sesi�n {\tt screen} en modo usuario, s�lo
tenemos que utilizar el flag {\tt -x}, al invocarlo.


\sectiontext{white}{black}{VARIAS IPs}
\hrule
\vspace{2mm}

La forma m�s sencilla de configurar varias IPs es disponiendo de
varios interfaces de red. Si no tenemos varias tarjetas podemos
utilizar {\em alias}. Esta es la forma de definirlos.

{\small
\begin{verbatim}
# ifconfig eth0:1 192.168.100.1 netmask 255.255.255.0
\end{verbatim}
}


El comando anterior crea un alias de nuestro interfaz eth0, llamado
eth0:1 con la nueva configuraci�n de red. Notad que esto es solo un
alias y no pod�is utilizarlo, por ejemplo, en las reglas de vuestro
firewall. 

Para configuraciones m�s complejas dispon�is de otras alternativas
como las VLAN (ya hablaremos de ellas en otra ocasi�n).



\sectiontext{white}{black}{PRODUCE VIDEOS DESDE TU WEBCAM}
\hrule
\vspace{2mm}

Solo tenemos que escribir esta l�nea:


{\small
\begin{verbatim}
$  mencoder -tv driver=v4l -ovc lavc -o movie.avi tv://
\end{verbatim}
}

Y tener instalados los paquetes apropiados (mplayer, mencoder, ffmpeg,
libavcodec). 


\colorbox{introcolor}{
\begin{minipage}{.9\linewidth}{
\textbf{\textsf{Env�a tus trucos}}

\vspace{1mm}

\textsf{Puedes enviarnos esos trucos que usas a diario para compartirlos con el resto de lectores a la direcci�n: }

\vspace{2mm}

\texttt{occams-razor@uvigo.es}
}
\end{minipage}
}

\raggedcolumns
\pagebreak

\vspace{6cm}
\end{multicols}

\clearpage
\pagebreak

